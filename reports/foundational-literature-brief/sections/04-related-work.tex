%!TEX root = ../main.tex

\section{Related Work: Typed Lua, Luau, TypeScript, and Teal}
The HM type system functions on the basis of pure type inference through constraint generation and unification \cite{Damas:1982:PTS:582153.582176}. However, modern programming languages have adopted more pragmatic approaches to type inference, especially when it concerns dynamically typed languages. This section explores some of the most notable implementations of type inference systems in dynamically typed languages, including Typed Lua, Luau, TypeScript, and Teal.

\subsection{Typed Lua}
Typed Lua aims to bring optional static types to the Lua programming language while preserving its dynamic nature \cite{Maidl:2014:TLO:2617548.2617553}. Its approach to implementing a type inference system for Lua is through the use of purely optional typing, as opposed to gradual typing. In Typed Lua, the default is to have no runtime type checking, solely relying on static analysis to catch errors during development. This approach allows developers to incrementally add type annotations to their code, enabling them to choose which parts of their code should be subject to static type checking. Typed Lua's type system is designed to be sound and consistent, ensuring that type errors are caught during development without introducing runtime overhead. However, this approach can lead to situations where type errors are not caught until runtime if developers choose not to add type annotations to certain parts of their code. Additionally, Typed Lua's implementation demonstrates that annotations are necessary to achieve soundness in the type system, as pure type inference is unable to maintain soundness and completeness while maintaining Lua's full feature set.

\subsection{Luau}
Luau is a high-performance variant of Lua developed by Roblox, primarily used for scripting within the Roblox platform \cite{Brown:2021:GOL:3486606.3486770}. Luau implements a gradual type system that allows developers to optionally add type annotations to their code. This approach enables developers to choose which parts of their code should be subject to static type checking, providing a balance between flexibility and safety. Luau's type system is designed to be sound and consistent, ensuring that type errors are caught during development without introducing runtime overhead. Additionally, Luau's type system includes features such as union types, intersection types, and type aliases, allowing for greater expressiveness and flexibility in defining types.

Luau's implementation demonstrates the effectiveness of gradual typing in dynamically typed languages, providing developers with the ability to incrementally adopt static typing as needed. The team behind Luau has described it to be a "type inference engine first and a type checker second", emphasizing the importance of type inference in their design. A unique aspect of Luau's type inference system, compared to a system such as Hindley-Milner, is its use of localizing type inference to specific scopes, allowing for more precise type inference in the presence of mutable state and dynamic features. In Luau's type system, a variable's type can be refined based on control flow analysis, allowing for more accurate type inference in different contexts within the same scope. One of the key limitations of Luau's implementation of a type system is its complete reliance on type annotations for all data structures and functions. A notorious example of this limitation is Luau's inability to infer the type of \verb|self| within methods of tables (objects) without an explicit type annotation. However, these limitations are the result of a pragmatic design choice to prioritize performance and usability over theoretical soundness and completeness that might be seen in a pure type inference system such as Hindley-Milner.

\subsection{Teal}
Teal is a statically typed dialect of Lua that is trans-piled into standard Lua code. In its implementation, it is entirely annotation-driven. Teal does not serve to provide Lua with a type inference system, but rather to provide a separate development experience for developers who wish to use a statically typed language on top of Lua. Do to this separation from Lua, Teal is able to implement a consistent static type system without the need to accommodate Lua's dynamic features. However, the drawback of this approach is that the dynamic nature of these features are not represented identically. Teal separates the use of tables to be either records (static structures with defined fields), maps (dynamic structures with key-value pairs), or arrays (ordered collections). This separation allows for a more consistent type system, but deviates from Lua's core design where tables are used as a single, flexible data structure that can serve all these purposes. Additionally, Teal requires explicit type annotations for all variables and expressions, which can lead to increased verbosity and reduced flexibility compared to Lua's dynamic typing. While Teal provides an effective static type system for Lua-like code, it does so but not only using an impure type inference system, but a combination of an annotation-driven type system that is entirely separate from native Lua.