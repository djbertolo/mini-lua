%!TEX root = ../main.tex

\section{Introduction: Type Inference for a Lua-like Language}

\subsection{The Rise of Dynamically Typed Languages}
Throughout the last few decades, programming languages that have a dynamic type system have gained significant popularity. Languages such as Python, JavaScript, Ruby and Lua have been widely adopted, with their popularity stemming from their focus on concise syntax, rapid prototyping capabilities, and that types are not explicitly declared. However, this flexibility for rapid prototyping and simplicity reveals its cost when a project exceeds the prototyping phase.

A dynamic type system refers to when type validation is performed at runtime instead of at compile time. This behavior can lead to runtime errors that could have been discovered and remedied during the development phase if a static type system were in place. Additionally, this class of runtime errors can be difficult to debug as their very nature means that they only manifest during execution, often only under specific conditions. This can lead to increased costs to development teams in the form of time spent debugging and fixing these issues, as well as potential delays in project timelines.

\subsection{Static Analysis as a Solution}
Static analysis allows for a powerful solution to the problems caused by the nature of dynamic type systems \cite{Aiken:1991:STI:99583.99611}. By analyzing code without requiring execution, static analysis tools can algorithmically verify type constraints and confirm the abidance of code to a set of type rules. This implementation of static analysis can help catch potential type-related errors early in the development process, reducing the likelihood of runtime errors and improving overall code quality, maintainability, and reliability. It forms a hybrid approach that combines the flexibility of dynamic typing with the safety and reliability of static typing.

\subsection{The Importance of Lua}
The subject of this research is the Lua programming language, a light-weight, high-level, and dynamically typed language. Due to its highly performant nature and ease of embedding, it is widely used in various domains, including video game development, embedded systems, and scripting within larger applications. Its popularity within the video game industry is particularly notable, with titles such as \emph{World of Warcraft}, \emph{Balatro}, and \emph{Angry Birds} utilizing Lua for scripting game logic and behavior. Additionally, a high-performance variant of Lua, Luau \cite{Brown:2021:GOL:3486606.3486770}, is the primary scripting language for the Roblox platform, a platform hosting millions of active users and one of the largest development ecosystems in the world. Given Lua's widespread adoption and its significant role in various applications, enhancing its type system through static analysis can have large benefits for both developers and users.

\subsection{Lua's Dynamism and Challenges for Type Inference}
Lua's dynamic nature presents a difficult challenge for static analysis, especially concerning type inference. Similar to many dynamically typed languages, Lua does not require explicit type annotations, allowing for variables to hold values of any type and change types at runtime. This flexibility is common amongst popular dynamically typed languages, such as Python and JavaScript. However, this very flexibility complicates the process of static analysis, especially static type inference \cite{Aiken:1991:STI:99583.99611}. The absence of explicit type information means that the static analysis tool must deduce types based on the code's structure and behavior.

This behavior can be difficult to properly analyze when taking into consideration the various dynamic features of Lua and common programming patterns used by developers. Features such as first-class functions, dynamic table structures \cite{Lin2015OperationalSF}, and core language operation redefinition through the use of features such as metatables \cite{Lin2015OperationalSF}, and a mutable global state that can be modified at runtime, all contribute to a severe increase in complexity for static analysis tools attempting to infer types. Additionally, common programming patterns used by developers, such as duck typing and dynamic code generation, further contributes to the challenge of accurately inferring types in Lua programs.

\subsection{MiniLua: A Lua-like Language to Research}
A static analyzer is unable to effectively predict the full range of possible behaviors given a piece of code due to the dynamic features and programming patterns used in Lua \cite{Lin2015OperationalSF}. To successfully implement a static type inference system for Lua, a strict set of coding standards and limitations would have to be imposed on developers to reduce the complexity and dynamism of the language and its runtime behavior. However, these limitations would likely debilitate developers' ability to effectively use the language for its intended purpose.

By instead exploring a Lua-like language that retains the core syntax and semantics of Lua, while simplifying or restricting some of its more dynamic features, it becomes possible to design a static type inference system that is both effective and practical. This research explores MiniLua, a subset of the Lua programming language that aims to retain the core syntax and semantics of Lua while simplifying or restricting some of its dynamic features. By focusing on MiniLua, this research aims to explore the design and implementation of a static type inference system that can effectively analyze Lua-like code that maintains a balance between flexibility and static analyzability.