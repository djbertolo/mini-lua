%!TEX root = ../main.tex

\section{Conclusion: How This Review Informs MiniLua's Design}
The review of existing literature on type inference systems for Lua-like languages has provided valuable insights that will inform the design of MiniLua's type inference system. The analysis of various approaches, including Typed Lua, Luau, and Teal, has highlighted the challenges and trade-offs involved in designing type inference systems for dynamically typed languages.

\subsection{Lessons to Apply to MiniLua's Design}
MiniLua's type inference system should be constraint-based and solved using unification. By using the concepts of the HM type system \cite{Damas:1982:PTS:582153.582176}, and borrowing its implementation in languages such as ML and Haskell, MiniLua can achieve soundness and completeness in its type inference system. The implementation of polymorphism should aim to follow the let-polymorphism approach, ensuring that polymorphic types are only introduced in a controller manner at let-bindings. This restriction aligns will with a subset of Lua as the declaration of variables through \verb|local| statements can be treated as let-bindings. Continuing with concepts from the HM type system, MiniLua will borrow the principle of value restriction from implementations such as Standard ML and OCaml. This principle ensures that only expressions that are syntactic values can be generalized to polymorphic types, preventing unsoundness in the presence of mutable state. By enforcing this restriction, MiniLua can maintain soundness in its type system while still allowing for polymorphism in a controlled manner.

\subsection{Pragmatic Decisions for Language Grammar}
To effectively implement a functional programming driven subset of Lua, certain pragmatic decisions must be made regarding the language grammar. MiniLua should aim to limit or restrict features that are heavily dynamic in nature, or that encourage imperative programming patterns. This includes limiting the use of dynamic table structures and global variables \cite{Lin2015OperationalSF}. By reducing the complexity introduced by these features, MiniLua can maintain a more predictable and analyzable codebase, allowing for more accurate type inference. Additionally, MiniLua should borrow from Teal's approach of separating tables into distinct data structures that cannot be interchanged. This separation allows for a more consistent type system, as each data structure can have its own defined type and behavior. By implementing these pragmatic decisions in the language grammar, MiniLua can effectively balance the need for flexibility and expressiveness with the requirements of a sound and complete type inference system.

\subsection{Pragmatic Decisions for Type System}
A common implementation choice is the use of gradual typing, as seen in Luau. The use of gradual typing contradicts the goal of implementing a pure type inference system based on Hindley-Milner. While it stands as a pragmatic choice to prioritize usability and performance, it introduces compromises that stray from the theoretical soundness and completeness of the type system \cite{Brown:2021:GOL:3486606.3486770}. MiniLua's type system should avoid the use of gradual typing, instead opting for a pure type inference system based on the Hindley-Milner type system. This decision aligns with the goal of achieving soundness and completeness in the type inference system, while still allowing for flexibility and expressiveness through the use of polymorphism. By avoiding gradual typing, MiniLua can maintain a more consistent and predictable type system, ensuring that type errors are caught during development without introducing runtime overhead. A gradual type system may be accepting of certain unsound behaviors for the sake of usability, but MiniLua's type system should prioritize theoretical soundness and completeness to provide a robust and reliable development experience.

MiniLua's type inference system should prioritize soundness and completeness over abidance to the reference, Hindley-Milner type system. While polymorphism is a core feature of the HM type system, MiniLua should only implement polymorphism where it can be achieved without compromising soundness. This should be demonstrated through the successful implementation of complete and sound monomorphic type inference before extending to polymorphic types. By prioritizing soundness and completeness, MiniLua can provide a reliable and predictable development experience, ensuring that type errors are caught during development without introducing runtime overhead.